\documentclass[aip,jcp,a4paper,reprint,onecolumn]{revtex4-1}

\usepackage[fleqn]{amsmath}
\usepackage{amssymb}
\usepackage[dvips]{graphicx}
\usepackage{color}
\usepackage{tabularx}
\usepackage{algorithm}
\usepackage{algorithmic}

\makeatletter
\makeatother

\newcommand{\recheck}[1]{{\color{red} #1}}
\newcommand{\redc}[1]{{\color{red} #1}}
\newcommand{\bluec}[1]{{\color{blue} #1}}
\newcommand{\vect}[1]{\textbf{\textit{#1}}}
\newcommand{\dd}{\textsf{d}}
\newcommand{\inv}{\textrm{inv}}
\newcommand{\hard}{\textrm{h}}
\newcommand{\soft}{\textrm{s}}
\newcommand{\vdw}{\textrm{vdw}}
\newcommand{\ele}{\textrm{ele}}
\newcommand{\dir}{\textrm{dir}}
\newcommand{\rec}{\textrm{rec}}
\newcommand{\corr}{\textrm{corr}}

\newcommand{\mh}{\mathcal H}
\newcommand{\eps}{\varepsilon}
\newcommand{\ml}{\mathcal L}
\newcommand{\me}{\mathcal E}
\newcommand{\mt}{\mathcal T}
\newcommand{\mo}{\mathcal O}
\newcommand{\mi}{\mathcal I}
\newcommand{\mc}{\mathcal C}
\newcommand{\proj}{\mathit\Pi}
\newcommand{\fwg}{{\mathcal A}}
\newcommand{\bwg}{{\mathcal B}}
\newcommand{\bsigma}{\boldsymbol\sigma}
\newcommand{\pathmeas}{d\mathcal P}
\newcommand{\dist}{\textrm{dist}}
\newcommand{\randomf}{d\vect w}


\begin{document}


\title{Report: Important sampling by the cross-entropy method}
\author{Han Wang}
\affiliation{Institute for Mathematics, Freie Universit\"at Berlin, Germany}
% \author{Christof Sch\"utte}
% \affiliation{Institute for Mathematics, Freie Universit\"at Berlin, Germany}
% \affiliation{Zuse Institute Berlin (ZIB), Germany}
% \affiliation{Institute for Mathematics, Freie Universit\"at Berlin, Germany}

\date{\today}

\begin{abstract}
\end{abstract}

\maketitle

\section{Theory}
We study the following stochastic process:
\begin{align}\label{eqn:sde}
  dx_t = b(x_t) dt + \sigma dw_t,
\end{align}
and are interested in the observable defined by
\begin{align}
  O = \mathbb E(F[x_t]) = \int F[x_t] \pathmeas[x_t],
\end{align}
where $\pathmeas[x_t]$ defines the probability measure of the
trajectories generated by process~\eqref{eqn:sde}.  We want to reduce
the variance of calculating observable $O$ by applying control
$c(x_t)$ to the system, i.e.
\begin{align}
  dx_t = b(x_t) dt + c(x_t) dt + \sigma dw_t,  
\end{align}
Now the probability measure of the trajectories is denoted by
$\pathmeas_c[x_t]$. The observable is calculated by
\begin{align}
  O = \int F[x_t] \frac{\pathmeas[x_t]}{\pathmeas_c[x_t]} \pathmeas_c[x_t]
\end{align}
By the Girsanov transformation, we have
\begin{align}
  \frac{\pathmeas[x_t]}{\pathmeas_c[x_t]} = 
  \exp \Big\{ - \frac 1\sigma \int_0^t c(x_s) dw_s - \frac1{2\sigma^2} \int_0^t c^2(x_s) ds\Big\}
\end{align}\\

\noindent
The optimal control that minimized the variance of calculting the observable is reached when
\begin{align}\label{eqn:min-cond-1}
  \pathmeas_c^\ast[x_t] =\textrm{C} F[x_t] \pathmeas[x_t]
\end{align}
The constant \textrm{C} is determined with the nomorlization relation.
The cross-entropy, as a measure of the distance between two
probability measures, is defined by
\begin{align}
  \dist(\pathmeas_1, \pathmeas_2)
  = \mathbb E_1 \Big[\log \Big(\frac{\pathmeas_1}{\pathmeas_2}\Big)\Big]
  = \int \log \Big(\frac{\pathmeas_1}{\pathmeas_2}\Big)\, \pathmeas_1
\end{align}
In order to calculate the optimal control $\pathmeas_c^\ast$, we
minmize the distance between the measure $\pathmeas_c$~and the
R.H.S.~of Eq.~\eqref{eqn:min-cond-1}, and savely neglecting the
constant in the equation:
\begin{align}
  \min_c \int  \log \Big( \frac{F[x_t]\pathmeas[x_t]}{\pathmeas_c[x_t]} \Big) F[x_t]\pathmeas[x_t],
\end{align}
which is equivalent to the problem
\begin{align}\label{eqn:max-1}
  \max_c \int \log \Big(\frac{\pathmeas_c[x_t]}{\pathmeas[x_t]} \Big) F[x_t]\pathmeas[x_t].
\end{align}
Computationally, we provide an intial guess for the control $c_0$, then we can change Eq.~\eqref{eqn:max-1} to
\begin{align}
  \max_c \int
  \log \Big(\frac{\pathmeas_c[x_t]}{\pathmeas_{c_0}[x_t]} \Big)
  F[x_t]
  \frac{\pathmeas[x_t]}{\pathmeas_{c_0}[x_t]}
  \,\pathmeas_{c_0}[x_t]
\end{align}
A step with the Girsanov transformation
\begin{align}
  \max_c \int
  \Big[\, \frac1\sigma\int_0^t\tilde c(x_s) dw_s - \frac 1{2\sigma^2} \int_0^t\tilde c^2(x_s) ds \,\Big]
  \,F[x_t]\,
  \exp
  \Big\{ -\frac1\sigma\int_0^t c_0(x_s) dw_s - \frac 1{2\sigma^2} \int_0^t c_0^2(x_s) ds \Big \}
  \,\pathmeas_{c_0}[x_t].
\end{align}
where we denote the increment of the control by
\begin{align}
  \tilde c = c - c_0.
\end{align}
We expand the control with respect to a set of base functions $\{\varphi_i\}$:
\begin{align}
  c = \sum_{i=0}^N w_i\varphi_i, \quad c_0 = \sum_{i=0}^N w_{0,i}\varphi_i, \quad \tilde c = \sum_{i=0}^N \tilde w_i\varphi_i,
\end{align}
We then have the optimization problem
\begin{align}
  \max \int F[x_t]
  \Big ( \sum_i \tilde w_i g_i - \frac12\sum_{ij} \tilde w_i\tilde w_j h_{ij} \Big)
  \exp \Big\{
  - \sum_i w_{0,i} g_i - \frac 12 \sum_{ij} h_{ij} w_{0,i}    w_{0,j} \Big\}
  \,\pathmeas_{c_0}[x_t]
\end{align}
with
\begin{align}\label{eqn:gi}
  g_i[x_t] & = \frac 1\sigma\int_0^t \varphi_i(x_s) dw_s \\\label{eqn:hij}
  h_{ij}[x_t] & = \frac1{\sigma^2} \int_0^t \varphi_i(x_s)  \varphi_j(x_s) ds
\end{align}
Now define
\begin{align}
  b_i & = \int F[x_t]\, g_{i}[x_t] \exp \Big\{
  - \sum_i w_{0,i} g_i - \frac 12 \sum_{ij} h_{ij} w_{0,i}    w_{0,j} \Big\}
  \,\pathmeas_{c_0}[x_t]\\
  A_{ij} & = \int F[x_t]\, h_{ij}[x_t] \exp \Big\{
  - \sum_i w_{0,i} g_i - \frac 12 \sum_{ij} h_{ij} w_{0,i}    w_{0,j} \Big\}
  \,\pathmeas_{c_0}[x_t]
\end{align}
Then the optimization problem is
\begin{align}
  \max_{\tilde w } -\frac 12 \tilde w^T A w + b^T \tilde w
\end{align}
To find the optimal control, we need only to solve the linear equation
\begin{align}
  A\tilde w = b, \quad \textrm{or}\quad A w = A w_0 + b
\end{align}
In practice, due to the statistical uncertainty, this equation may not
be solve exactly, so we need a iterative process:
\begin{align}
  w_{i+1} = w_i + A^{-1}b
\end{align}
After we have got the converged control $w_i^\ast$,
i.e.~the optimal one, we have
the observable calculated by the important sampling formula:
\begin{align}\label{eqn:import-samp}
  O = \int F[x_t]\,
  \exp\Big\{
  -\sum_iw^\ast_i g_i -\frac12\sum_{ij}w^\ast_iw^\ast_jh_{ij}
  \Big\}\,
  \pathmeas_{c^\ast}[x_t]
\end{align}


\subsection{Variance reduction}

We firstly consider the change of variable due to the rule:
\begin{align}
  y = f(x).
\end{align}
The observable is then defined with respect to the new variable $y$
instead of $x$:
\begin{align}
  I = \int F[y_t] \pathmeas[y_t]
\end{align}
where $\pathmeas[y_t]$ is the probability measure of the trajectories
governed by
\begin{align}
  dy_t = \hat b(y_t) dt + \hat\sigma (y_t) dw_t
\end{align}
where $\hat b(y_t) = [\nabla f b + \frac12\sigma^2\nabla^2f]$ and
$\hat\sigma(y_t) = \nabla f\sigma$.
Here we assume that some DOFs of $y$ are independent from others, i.e.
\begin{align}
  y = [y^1, y^2].
\end{align}
For instance, $y^1$ can be translational DOFs of a molecule, while
$y^2$ are the internal DOFs that describe the molecular configuration.
We usually have
\begin{align}
  dy^1_t &= \hat b^1dt + \hat\sigma^{11} dw^1_t + \hat\sigma^{12}dw^2_t\\
  dy^2_t &= \hat b^2dt + \hat\sigma^{21} dw^1_t + \hat\sigma^{22}dw^2_t
\end{align}
Note that both $\hat b$ and $\hat \sigma$ can be $y$ dependent.
We assume the coupling between the DOFs $y^1$ and $y^2$ is weak,
then, the probability measure of the trajectories therefore has
\begin{align}  
\quad \pathmeas[y_t] \approx \pathmeas[y^1_t]\pathmeas[y^2_t].
\end{align}
The control should also be designed on the variable $y^2$
\begin{align}
  dy^2_t &= \hat b^2dt + \hat c(y^2_t)dt + \hat\sigma^{21} dw^1_t + \hat\sigma^{22}dw^2_t
\end{align}
% With the optimal control, the important sampling for the observable is
% \begin{align}
%   O = \int F[y_t]
%   \exp\Big\{
%   -\sum_i w_i^\ast\int_0^t \hat\sigma^{-1}\hat\varphi_i(y_t) dw^2_s
%   - \frac12 \sum_{ij}w^\ast_iw^\ast_j \int_0^t\hat\sigma^{-2}\hat\varphi_i(y_t)\hat\varphi_j(y_t) ds
%   \Big\} \pathmeas_{c^\ast}[y_t]
% \end{align}
We assume that the instantaneous observable $F$ depends only on DOFs $y^2$, then
the important sampling calculation of the observable is
\begin{align}\label{eqn:eqn-diff}
  O &= \int F[y^2_t]
  \exp\Big\{
  -\sum_i w_i^\ast\int_0^t \hat\sigma^{-1}\hat\varphi_i(y^2_t) dw^2_s
  - \frac12 \sum_{ij}w^\ast_iw^\ast_j \int_0^t[\hat\sigma^{-1}\hat\varphi_i(y^2_t)] [\hat\sigma^{-1}\hat\varphi_j(y^2_t)] ds
  \Big\} \pathmeas_{c^\ast}[y^1_t]  \pathmeas_{c^\ast}[y^2_t] 
\end{align}
If $y^1$ are the translational DOFs, $y = f(x)$ can be a linear
transformation~\cite{littlejohn1997gauge}, then one can show that $\hat\sigma$ is independent
with $y$,
% \begin{align}
%   \hat\sigma^{11} = \hat\sigma^{11}(y^1), \quad
%   \hat\sigma^{22} = \hat\sigma^{22}(y^2), \quad
%   \hat\sigma^{12} = \hat\sigma^{21} = 0
% \end{align}
then the integral over $\pathmeas_{c^\ast}[y^1_t]$ in
Eq.~\eqref{eqn:eqn-diff} is trival, which implies the translational
DOFs contribute trivally when one compute $g_i$ (Eq.~\eqref{eqn:gi})
in coordinate $x$, and the translational contribution in $dw_t$ should
be manually removed. \recheck{For the rotational DOFs, the couplping
  is not simple... how to prove...}

% \subsection{Variance reduction}

% The calculation of the observable, i.e.~Eq.~\eqref{eqn:import-samp} can be
% optimized by a variance reduction technique, which is computationally
% very cheap, and can increase the accuracy significantly. We assume
% that the noise $dw_t$ can be divided into two parts,
% \begin{align}
%   dw_s = f(dw_s) + [\,dw_s - f(dw_s)\,].
% \end{align}
% The first part is statistically independent with the functional
% $F[x_t]$ and the second part. For simplicity, we assume this splitting
% is the same for all the base functions $\varphi_i$. It can be, take the
% butane system for example, the translational component of $dw_s$, then
% it is independent with $F[x_t]$, if the latter is the dihedral angle
% of the butane.  We then have the splitting of $g_i$:
% \begin{align}
%   g_i[x_t]
%   = g_i^i[x_t] + g_i^d[x_t]
%   = \frac1\sigma\int_0^t\varphi_i(x_s) f(dw_s)
%   + \frac1\sigma\int_0^t\varphi_i(x_s) [1 - f(dw_s)]
% \end{align}
% We want to prove that
% \begin{align}\label{eqn:o-approx}
%   O \approx \int F[x_t]
%   \exp\Big\{
%   -\sum_iw^\ast_i g^d_i -\frac12\sum_{ij}w^\ast_iw^\ast_jh_{ij}
%   \Big\}\,
%   \pathmeas_{c^\ast}[x_t]    
% \end{align}
% We start from Eq.~\eqref{eqn:import-samp}:
% \begin{align}
%   O = \int F[x_t]
%   \exp\Big\{
%   -\sum_i w^\ast_ig^i_i
%   -\sum_iw^\ast_i g^d_i -\frac12\sum_{ij}w^\ast_iw^\ast_jh_{ij}
%   \Big\}\,
%   \pathmeas_{c^\ast}[x_t]  
% \end{align}
% If we further assume that $w_i^\ast$ is small, the Taylor expansion to
% the first order gives
% \begin{align}
%   O \approx \int F[x_t]
%   \Big( 1 - \sum_i w_i^\ast g_i^i   -\sum_iw^\ast_i g^d_i\Big)
%   \pathmeas_{c^\ast}[x_t]    
% \end{align}
% Take the term with $g_i^i$, it gives
% \begin{align}
%   w_i^\ast \frac1\sigma
%   \int F[x_t] 
%   \Big[ \int_0^t\varphi_i(x_s)f(dw_s)
%   \Big]
%   \pathmeas_{c^\ast}[x_t].
% \end{align}
% Due to the statistical independency, it is equal to
% \begin{align}\label{eqn:tmp28}
%   w_i^\ast \frac1\sigma
%   \int
%     \Big[ \int_0^t\varphi_i(x_s)f(dw_s)
%   \Big]
%   \pathmeas_{c^\ast}[x_t]
%   \cdot
%   \int F[x_t] 
%   \pathmeas_{c^\ast}[x_t] = 0, 
% \end{align}
% because the first expectation value in Eq.~\eqref{eqn:tmp28} is equal to 0.
% Therefore, Eq.~\eqref{eqn:o-approx} holds.
% % If we further assume that $w_i^\ast g_i^i$ is small, the Taylor expansion of
% % the first exponent gives
% % \begin{align}
% %   O \approx \int F[x_t]
% %   \Big( 1 - \sum_i w_i^\ast g_i^i \Big)
% %   \exp\Big\{
% %   -\sum_iw^\ast_i g^d_i -\frac12\sum_{ij}w^\ast_iw^\ast_jh_{ij}
% %   \Big\}\,
% %   \pathmeas_{c^\ast}[x_t]    
% % \end{align}
% % Take one of the term in the first sum for example
% % \begin{align}
% %   w_i^\ast \frac1\sigma
% %   \int F[x_t] 
% %   \Big[ \int_0^t\varphi_i(x_s)f(dw_s)
% %   \Big]
% %   \exp\Big\{
% %   -\sum_iw^\ast_i g^d_i -\frac12\sum_{ij}w^\ast_iw^\ast_jh_{ij}
% %   \Big\}\,
% %   \pathmeas_{c^\ast}[x_t],
% % \end{align}
% % which is equal to
% % \begin{align}\label{eqn:tmp28}
% %   w_i^\ast \frac1\sigma
% %   \int
% %     \Big[ \int_0^t\varphi_i(x_s)f(dw_s)
% %   \Big]
% %   \pathmeas_{c^\ast}[x_t]
% %   \cdot
% %   \int F[x_t] 
% %   \exp\Big\{
% %   -\sum_iw^\ast_i g^d_i -\frac12\sum_{ij}w^\ast_iw^\ast_jh_{ij}
% %   \Big\}\,
% %   \pathmeas_{c^\ast}[x_t] = 0, 
% % \end{align}
% % because the first expectation value in Eq.~\eqref{eqn:tmp28} is equal to 0.
% % Therefore, Eq.~\eqref{eqn:o-approx} holds.



\subsection{Practical details of the variance reduction}

We denote the four carbon atoms on the molecules by indexes 0--4. The
position of them are donted by $\vect r_0$, $\vect r_1$, $\vect r_2$
and $\vect r_3$. The three bonds connecting the atoms are denoted by
\begin{align}
  \vect r_{01} = \vect r_1 - \vect r_0, \quad
  \vect r_{12} = \vect r_2 - \vect r_1, \quad
  \vect r_{23} = \vect r_3 - \vect r_2
\end{align}
We let $\vect e_{12} = \vect r_{12} / \vert\vect r_{12}\vert$.  The
normal vector of the planes formed by atom 0, 1, 2 and 1, 2, 3 are
denoted by $\vect u_0$ and $\vect u_1$, respectively. They are
calculated by
\begin{align}
  \vect u_0 = \frac{\vect r_{01} \times\vect r_{12}}{\vert \vect r_{01} \times\vect r_{12}\vert},\quad
  \vect u_1 = \frac{\vect r_{12} \times\vect r_{23}}{\vert \vect r_{12} \times\vect r_{23}\vert}
\end{align}
The dihedral angle is calculated by
\begin{align}
  \phi = \arccos\Big(\frac{\vect u_0\cdot\vect u_1}{\vert \vect u_0\vert\,\vert\vect u_1\vert}\Big)
\end{align}
The random force exerting on the four atoms are denoted by
$\randomf_0$, $\randomf_1$, $\randomf_2$ and $\randomf_3$, respectively. We
further assume that $\vect u_0$ and $\vect u_1$ are not linearly
dependent, which is true unless $\phi = 0^\circ$ or $180^\circ$.  It
can be shown that the control force exerting on each atom is
perpendicular to $\vect e_{12}$, and only the inner produce of the
control force and the random force matters, therefore, it is safe to
further assume that the $\vect e_{12}$ compenent of the random force
$\randomf_i$ vanishes. Moreover, it can be shown that the control force
$\randomf_0$ and $\randomf_3$ are perpendicular to the plane of 0, 1, 2
and 1, 2, 3, respectively, so $\randomf_0$ does not have $\vect u_1$
component and $\randomf_3$ does not have $\vect u_0$ component.  We have the
decomposition of the random force:
\begin{align}
  \randomf_0 &= f_0 \vect u_0\\
  \randomf_1 &= f_{10}\vect u_0 + f_{11}\vect u_1\\
  \randomf_2 &= f_{20}\vect u_0 + f_{21}\vect u_1\\
  \randomf_3 &= f_3 \vect u_1
\end{align}

Since the translational DOFs are note ralevant to the conformational
transition of the molecules, therefore, the translational contribution
of the random force should be removed when computing the integral
Eq.~\eqref{eqn:gi}. That means the random force should satisfy
\begin{align}\label{eqn:cond-trans-0}
  f_0 + f_{10} + f_{20} &= 0\\\label{eqn:cond-trans-1}
  f_{11} + f_{21} + f_3 &= 0
\end{align}
They can be achieved by the following transforms:
\begin{align}
  f_0 &= f_0 - \frac{f_0 + f_{10} + f_{20}}3,\quad 
  f_{10} = f_{10} - \frac{f_0 + f_{10} + f_{20}}3,\quad 
  f_{20} = f_{20} - \frac{f_0 + f_{10} + f_{20}}3\\
  f_{11} &= f_{11} - \frac{f_{11} + f_{21} + f_3}3,\quad 
  f_{21} = f_{21} - \frac{f_{11} + f_{21} + f_3}3,\quad 
  f_{3} = f_{3} - \frac{f_{11} + f_{21} + f_3}3
\end{align}

We assume that the origin of the coordinate is the COM of the
molecule, then the torque of on the molecule with repect to the COM is
calculated by:
\begin{align}
  \boldsymbol\tau
  =
  f_0 \vect r_0\times \vect u_0 + 
  f_{10} \vect r_1\times \vect u_0 + 
  f_{11} \vect r_1\times \vect u_1 + 
  f_{20} \vect r_2\times \vect u_0 + 
  f_{21} \vect r_2\times \vect u_1 + 
  f_3 \vect r_3\times \vect u_1 
\end{align}
By using conditions Eq.~\eqref{eqn:cond-trans-0} and
\eqref{eqn:cond-trans-1}, we have
\begin{align}
  \boldsymbol\tau =
  (f_{10}\vect r_{01} + f_{20} \vect r_{02})\times \vect u_0 +
  (f_{11}\vect r_{31} + f_{21} \vect r_{32})\times \vect u_1
\end{align}
The rotational with respect to the axis $\vect e_{12}$ is independent with
the conformational transition of the molecule, therefore, we let
$\vect e_{12}\cdot \boldsymbol\tau  = 0$, which means
\begin{align}
  f_{10} \vect e_{12}\cdot(\vect r_{01}\times\vect u_0) + 
  f_{20} \vect e_{12}\cdot(\vect r_{02}\times\vect u_0) + 
  f_{11} \vect e_{12}\cdot(\vect r_{31}\times\vect u_1) + 
  f_{21} \vect e_{12}\cdot(\vect r_{32}\times\vect u_1)  = 0
\end{align}
which is equavalent to
\begin{align}\label{eqn:cond-rot-e12-0}
  f_{10} \vect r_{01}\cdot(\vect u_0\times\vect e_{12}) + 
  f_{20} \vect r_{02}\cdot(\vect u_0\times\vect e_{12}) + 
  f_{11} \vect r_{31}\cdot(\vect u_1\times\vect e_{12}) + 
  f_{21} \vect r_{32}\cdot(\vect u_1\times\vect e_{12}) = 0
\end{align}
It is easy (almost obvious) to show that
\begin{align}
  \vect r_{01}\cdot(\vect u_0\times\vect e_{12}) = \vect r_{02}\cdot(\vect u_0\times\vect e_{12}) & = h_0, \\
  \vect r_{31}\cdot(\vect u_1\times\vect e_{12}) = \vect r_{32}\cdot(\vect u_1\times\vect e_{12}) & = h_3,
\end{align}
where $h_0$ ($h_3$) is the distance between atom 1 (atom 3) and vector
$\vect e_{12}$. The condition Eq.~\eqref{eqn:cond-rot-e12-0} becomes
\begin{align}\label{eqn:tmp49}
  (f_{10} + f_{20}) h_0 + (f_{11} + f_{21}) h_3 = 0
\end{align}
or
\begin{align}
  f_0h_0 + f_3h_3 = 0
\end{align}
% With these conditions, it can be easily show that
If we require the $\vect e_{12}$ conponent of relative torque of plane
012 to 123 (controls the change of dihedral angle $\phi$) is constant, then we have
\begin{align}\label{eqn:tmp51}
  (f_{10} + f_{20}) h_0 - (f_{11} + f_{21}) h_3 = C
\end{align}
or
\begin{align}
  f_0h_0 - f_3h_3 = -C
\end{align}
where $C$ is a constant.
We further calculate
\begin{align}\nonumber
  (\vect u_1 + \vect u_0) \cdot\boldsymbol\tau
  &=
  f_{10} \vect r_{01}\cdot(\vect u_0\times\vect u_{1}) + 
  f_{20} \vect r_{02}\cdot(\vect u_0\times\vect u_{1}) + 
  f_{11} \vect r_{31}\cdot(\vect u_1\times\vect u_{0}) + 
  f_{21} \vect r_{32}\cdot(\vect u_1\times\vect u_{0})  \\
  &  =
  (f_{10} \vect r_{01} + f_{20} \vect r_{02} - f_{11} \vect r_{31} - f_{21} \vect r_{32}) \cdot \vect e_{12} \\\nonumber
  (\vect u_1 - \vect u_0) \cdot\boldsymbol\tau
  &=
  f_{10} \vect r_{01}\cdot(\vect u_0\times\vect u_{1}) + 
  f_{20} \vect r_{02}\cdot(\vect u_0\times\vect u_{1}) - 
  f_{11} \vect r_{31}\cdot(\vect u_1\times\vect u_{0}) -
  f_{21} \vect r_{32}\cdot(\vect u_1\times\vect u_{0})  \\
  &  =
  (f_{10} \vect r_{01} + f_{20} \vect r_{02} + f_{11} \vect r_{31} + f_{21} \vect r_{32}) \cdot \vect e_{12} 
\end{align}
Requiring the torque on $\vect u_1 + \vect u_0$ and $\vect u_1 + \vect
u_0$ directions vanish, we have conditions:
\begin{align}\label{eqn:tmp55}
  f_{10} d_{01} + f_{20} d_{02} &= 0\\\label{eqn:tmp56}
  f_{11} d_{31} + f_{21} d_{32} &= 0
\end{align}
where $d_{01}$, $d_{02}$, $d_{31}$ and $d_{32}$ are the length of
projections (absolute value) of $\vect r_{01}$, $\vect r_{02}$, $\vect
r_{31}$ and $\vect r_{32}$ on vector $\vect e_{12}$.

By solving Eqs.~\eqref{eqn:tmp49}, \eqref{eqn:tmp51},
\eqref{eqn:tmp55} and \eqref{eqn:tmp56}, we finally have
\begin{align}
  &f_{10} = \frac{C d_{02}}{2h_0 d_{12}}, \quad f_{20} = -\frac{C d_{01}}{2h_0 d_{12}},\quad
  f_{11} = \frac{C d_{32}}{2h_3 d_{12}}, \quad f_{21} = -\frac{C d_{31}}{2h_3 d_{12}},\quad\\
  &f_{0} = - \frac{C d_{02}}{2h_0 d_{12}} + \frac{C d_{01}}{2h_0 d_{12}}, \quad
  f_{20} = - \frac{C d_{32}}{2h_3 d_{12}} + \frac{C d_{31}}{2h_3 d_{12}}
\end{align}


\section{Results}
The numerical results are summarized in the tables.
\begin{table}[th]
  \centering
  \caption{In solvent, variance reduction}
  \begin{tabular*}{0.8\textwidth}{@{\extracolsep{\fill}}lcccrr}
    \hline\hline
    $b$ & $P (\tau \leq b)$ & error & Var & accel. & Traj. Usage \\\hline
    0.1 & $4.61\times 10^{-5}$ & $1.11\times 10^{-5}$ & $2.45\times10^{-6}$ &18.8 & 0.7\%\\
    0.2 & $9.03\times 10^{-4}$ & $0.77\times 10^{-4}$ & $1.19\times10^{-4}$ & 7.6 &10.0\%\\
    0.5 & $6.96\times 10^{-3}$ & $0.34\times 10^{-3}$ & $2.21\times10^{-3}$ & 3.1 &11.9\%\\
    1.0 & $1.68\times 10^{-2}$ & $0.06\times 10^{-2}$ & $6.30\times10^{-3}$ & 2.7 &13.4\%\\
    \hline\hline
  \end{tabular*}
  \caption{In solvent, brute force}
  \begin{tabular*}{0.8\textwidth}{@{\extracolsep{\fill}}lcccrr}
    \hline\hline
    $b$ & $P (\tau \leq b)$ & error & Var & accel. & Traj. Usage \\\hline
    0.2 & $9.00\times 10^{-4}$ & $2.12\times 10^{-4}$ & $8.99\times10^{-4}$ & 1.0 & 0.1\%\\
    0.5 & $7.40\times 10^{-3}$ & $0.61\times 10^{-3}$ & $7.35\times10^{-3}$ & 1.0 & 0.7\%\\
    1.0 & $1.71\times 10^{-2}$ & $0.09\times 10^{-2}$ & $1.68\times10^{-2}$ & 1.0 & 1.7\%\\
    \hline\hline
  \end{tabular*}
\end{table}

\begin{table}[th]
  \centering
  \caption{Free molecule}
  \begin{tabular*}{0.8\textwidth}{@{\extracolsep{\fill}}lcccrr}
    \hline\hline
    $b$ & $P (\tau \leq b)$ & error & Var & accel. & Traj. Usage \\\hline
    0.1 & $6.38\times 10^{-5}$ & $1.75\times 10^{-5}$ & $1.23\times10^{-5}$ & 5.2 & 0.5\%\\
    0.2 & $1.59\times 10^{-3}$ & $0.17\times 10^{-3}$ & $4.79\times10^{-4}$ & 3.3 & 9.3\%\\
    0.5 & $8.27\times 10^{-3}$ & $0.49\times 10^{-3}$ & $4.75\times10^{-3}$ & 1.7 &11.8\%\\
    1.0 & $2.21\times 10^{-2}$ & $0.09\times 10^{-2}$ & $1.49\times10^{-2}$ & 1.5 &14.3\%\\
    \hline\hline
  \end{tabular*}
  \label{tab:tmp1}
  \caption{Free molecule after variance reduction}
  \begin{tabular*}{0.8\textwidth}{@{\extracolsep{\fill}}lcccrr}
    \hline\hline
    $b$ & $P (\tau \leq b)$ & error & Var & accel. & Traj. Usage \\\hline
    0.1 & $5.76\times 10^{-5}$ & $0.70\times 10^{-5}$ & $3.73\times10^{-6}$ &15.6 & 1.2\%\\
    0.2 & $1.35\times 10^{-3}$ & $0.09\times 10^{-3}$ & $1.55\times10^{-4}$ & 8.7 &17.1\%\\
    0.5 & $9.32\times 10^{-3}$ & $0.40\times 10^{-3}$ & $3.27\times10^{-3}$ & 2.9 &16.9\%\\
    1.0 & $2.19\times 10^{-2}$ & $0.07\times 10^{-2}$ & $1.03\times10^{-2}$ & 2.1 &18.5\%\\
    \hline\hline
  \end{tabular*}
  \caption{Two points fixed molecule}
  \begin{tabular*}{0.8\textwidth}{@{\extracolsep{\fill}}lcccrr}
    \hline\hline
    $b$ & $P (\tau \leq b)$ & error & Var & accel. & Traj. Usage \\\hline
    0.1 & $2.23\times 10^{-6}$ & $1.49\times 10^{-6}$ & $1.26\times10^{-7}$ & 17.7 & 0.1\%\\
    0.2 & $3.27\times 10^{-4}$ & $0.27\times 10^{-4}$ & $2.42\times10^{-5}$ & 13.5  & 43.0\%\\
    0.5 & $4.58\times 10^{-3}$ & $0.18\times 10^{-4}$ & $6.55\times10^{-4}$ & 7.0 & 68.6\%\\
    1.0 & $1.21\times 10^{-2}$ & $0.07\times 10^{-2}$ & $2.19\times10^{-3}$ & 5.5  & 47.5\%\\
    \hline\hline
  \end{tabular*}
  \caption{Three points fixed molecule}
  \begin{tabular*}{0.8\textwidth}{@{\extracolsep{\fill}}lcccrr}
    \hline\hline
    $b$ & $P (\tau \leq b)$ & error & Var & accel. & Traj. Usage \\\hline
    % 0.1 & $2.23\times 10^{-6}$ & $1.49\times 10^{-6}$ & $1.26\times10^{-7}$ & 17.7 & 0.1\%\\
    0.2 & $5.60\times 10^{-5}$ & $0.35\times 10^{-5}$ & $2.41\times10^{-7}$ & 231.4  & 44.8\%\\
    0.5 & $3.42\times 10^{-3}$ & $0.07\times 10^{-3}$ & $6.60\times10^{-5}$ & 49.3 & 74.0\%\\
    1.0 & $8.72\times 10^{-3}$ & $0.29\times 10^{-3}$ & $1.73\times10^{-3}$ & 5.0  & 82.2\%\\
    \hline\hline
  \end{tabular*}
\end{table}





\newpage
\section*{References}
\bibliography{ref}{}
\bibliographystyle{unsrt}




\end{document}
