\documentclass[]{revtex4-1}

\usepackage[fleqn]{amsmath}
\usepackage{amssymb,amsthm}

\newcommand{\fwg}{{\mathcal A}}

\begin{document}

\subsection*{The scientific issues from referee 1}

\emph{
The abstract and the general tones of the introduction and conclusion do not make justice to
the content of the paper. Whether the approach proposed by the authors generalizes standard
linear response theory as described in Section 2.1 is (at best) unclear to me. In fact, I think
that most of the material from Section 2.1 could be omitted. If the authors decide to keep the
material, they should more precisely compare (4) and (10), and show how (10) generalizes (4).
}

\emph{
One trouble is that I do not understand (4). As far as I understand, the equation before (4)
is (denoting by $\fwg$ the generator)
\begin{align*}
  \textbf{E}_{\rho^\varepsilon}[f] = \int f\rho^\varepsilon = \int e^{t\fwg} f\rho_0 = \mathbb{E}_{\rho_0}(f(X_t))
\end{align*}
where the latter expectation is over all initial conditions $X_0\sim\rho_0$ and over all realizations of
the dynamics (1). Is $f$ a functional of the trajectory or a given static observable? Is $t$ fixed
or is it taken to the limit $t \rightarrow+\infty$? Is the expectation $\textbf{E}_{\rho^\varepsilon}$
an expectation over all initial
conditions and all realizations of the dynamics? Assuming that the limit $t \rightarrow\infty$ is taken,
\begin{align*}
  \textbf{E}_{\rho^\varepsilon}[f] = \int f\mu_\varepsilon
\end{align*}
where $\mu_\varepsilon = \lim_{t\rightarrow+\infty}\rho^{\varepsilon}(t)$ is the (assumed) unique invariant measure of the dynamics. To
obtain (4), a linear response result on $\mu_{\varepsilon}$ is needed, as given by (3.19) in Ref. [5]. I would however rather write $\textbf{E}_{\rho_0}$ on the right hand side of (4), and emphasize that this is a static
average (no time dependence; compare with (10)). The authors should also make clear that
$\rho_0 (t) = e^{t\fwg^\ast} = \rho_0$ is in fact independent of $t$. Unless the authors mean something else with
$\rho_0$? (see the last equation of Section 2.1).
}

\emph{
I am also not very satisfied by the discussion at the beginning of Section 2.2. It seems
that the authors say that they can tackle cases uncovered by standard linear response theory
-- namely degeneracy of the invariant measure, absence of spectral gap, etc. But again, their
approach is a finite time perturbation theory, so I do not see why the discussion on the
limitation of standard perturbation theory (on invariant measures) is relevant.
}\\

We have rewritten Sec. 2.1, and restricted the discussion of the
equilibrium response theory (3) only in the finite-time
horizon, so that it is comparable to the nonequilibrium response theory (10), which is also
developed in the finite-time horizon. Our work
can be treated as a
generalization of the equilibrium response theory in the sense that the
reference process is not assumed to be in equilibrium.
Being more general, our theory does not assume the existence of nonequilibrium steady state.
The introduction and conclusion are revised accordingly.

The notations we used for the probability measures (both for
trajectory ensemble and initial condition ensemble) were misleading.
In the updated manuscript, we adopt the following convention:
$\rho^\varepsilon = \rho^\varepsilon(x,t)$ is now used only in the equilibrium
response theory part (Sec.~2.1). It denotes the probability density of
perturbed stochastic process $X_t$ (governed by SDE (1))
being at position $x$ at time $t$,
and is also the solution to the corresponding perturbed Fokker-Planck
equation. 
$\rho^0 = \rho^0(x, t)$
is the solution to the unperturbed Fokker-Planck that corresponds
to SDE (2). $\rho_\infty$ denotes the unique invariant measure of the
unperturbed Langevin dynamics.
If the initial distribution is assumed to be
$\rho^0(x, 0) = \rho_\infty$, it is obvious
that $\rho^0(x, t) = \rho_\infty(x)$ for any $t\geq 0$.
In the section 2.2 of nonequilibrium response theory, we adopt the new notation
$P_x^\varepsilon$ that denotes the trajectory probability density for the
perturbed process that starts with initial condition $X_0 = x$.
Correspondingly, $P_x^0$ denotes the probability density of
unperturbed trajectories with initial condition of $x_0 = x$.
We further
comment on the fact that if one starts the nonequilibrium process 
from a certain initial distribution (rather than the above fixed point
initial condition),
the response formula (10) can
be straightforwardly extend by averaging both sides of the equation over
the initial probability measure.
% We remind the reader about the difference
% between $P_x^\varepsilon$ and $\rho^\varepsilon$ under Eq.(10).
Under this setting, the $\phi$ on the r.h.s.~of (10) is understood
as a functional of the trajectory $(x_t)_{0\leq t\leq T}$. 
\\


% With the improved manuscript, we believe the referee is now clear
% on the fact 
% that do NOT want to discuss the infinite-time scenario, 
% the degeneracy of the inital distribution, nor the existence and uniqueness
% of the staionary probability distribution in this paper.
% In our finite-time nonequilibrium response theory,
% we do NOT presume neither the equilibrium of the system,
% nor the exsistence of a invarient (nor staionary) measure.
% % In our finite time setting,
% % The nonequilibrium dynamics is fully discribed by the trajectory
% % probability density $P^\varepsilon_x$ and the initial distribution
% % (provided that these distributions are well-defined), and the
% % Girsanov transformation used to derive the response theory
% % is valid under mild restrictions.
% % We do not touch the issue of the inifite time convergence of the probability
% % distribution.
% Therefore, when compareing the finite-time  equilibrium response theory with our
% nonequilibrium response theory, we conclude that our research overcomes the
% limitations 
% is a generalization of the equilibrium version.

\emph{
  The auxiliary process $\xi_t$ is not unique. This should be emphasized. Is there an optimal
choice, allowing for instance to reduce the variance of the estimator for the linear response
(e.g. (10) or (17))?
}\\

% As far as we concerned, we do not have a clear answer to this question
% in the most general case.
In this paper, when we deal with  the Langevin dynamics, 
the friction
matrix is positive definite, so the auxiliary process $\xi_t$ is uniquely
solvable.
If the matrix in front of the noise does not have full rank, then
there are some components of the driving Brownian motion in the SDE
that are uncorrelated from $X_t$, in which case the extra terms are
independent of $\varphi$ that depends only on $(X_t)$. Hence the
corresponding 1st and 2nd order linear response expressions vanish
because of the Martingale property $\textbf{E}[\int \xi^{extra}\cdot dB^{extra}]=0$
and the Ito isometry $\textbf{E}[(\int \xi^{extra}\cdot dB^{extra})^2]= \textbf{E} [\int
|\xi^{extra}|^2dt]$.
\\

\emph{
Why do the authors use a gradient method, rather than, say, a
conjugate gradient? How is the increment $\tau_n$ chosen in last equation of
page 8: fixed or optimized by a line search for instance? This is
never mentioned... Besides, I do not understand the final discussion
at the end of Section 3. I understand (22) (except for the above
remarks on who is $\rho^0$ , and whether this expectation means generating
initial conditions according to $\rho^0 (0, q, p)$ and evolving according to
some reference dynamics such as (18)). But I do not understand the
discussion on the feedback controls. In particular, how do the authors
make sure that the initial conditions are always distributed according
to the same measure? Or do the distributions of these initial
conditions convergence to some (a priori unknown) measure?
}\\

In the second numerical example (Sec.4.2), our aim is to verify the
application of nonequilibrium response theory (Eq.(22)) to the
optimization problem, but not to dig out the most efficient numerical
method for solving this optimization problem.  We prefer the
gradient descent method with a fixed increment, because firstly it
works and converges in only 22 steps of iterations. Secondly,
it is simple and easy to implement. Therefore, the gradient descent
method is eligible to illustrate the power of the response theory, and
at the same time keeps the story as simple as possible.

The confusing notations used in the old manuscript was clarified
in the updated manuscript, see the previous discussions.
The initial condition of the optimization problem is subject to
a prescribed distribution that is invariant with respect
to steps of iterations. The form of the initial distribution
depends on how the question is raised, for example, in Example 2,
we want to  calculate the optimal way of driving the
equilibrium distribution towards the right well of the potential.
Therefore, the initial distribution is naturally chosen as the
equilibrium distribution $\rho_\infty$.
At each step of iteration, we numerically solve the
equilibrium Langevin equation,
and take the configurations along the trajectory as initial conditions
for the nonequilibrium simulation.
These discussions have been added to the manuscript.
\\

\emph{ More details on the numerical simulations are needed to make
  them reproducible: which integrator is used to approximate the time
  evolution, what is the time step, what is the number of realizations
  M in (17), how are the initial conditions sampled (iid or
  subsampling of a long Langevin trajectory), etc. These elements may
  be trivial/standard, and I certainly do not want the authors to
  write more than a short paragraph about it; but I certainly want to
  read this paragraph. Moreover, on page 13, the optimization
  procedure may be documented more precisely (number of realizations,
  convergence of the estimator for the derivative and statistical
  error, etc). In the central finite difference, is some variance
  reduction technique used? Such as using the same realization of the
  random noise for the two dynamics with different forcings (a control
  variate method, as in J. B. Goodman and K. K. Lin, Coupling control
  variates for Markov chain Monte Carlo, J. Comput. Phys., 228,
  (2009), 7127–7136).  From what I understand, I would say ``yes''.  }\\

We have added all detailed information the referee asked to the updated
manuscript. Regarding the finite difference scheme, we did not used
the technique described in the literature by Goddman and Lin.
However, we have numerical checked the accuracy of the finite
difference scheme, and its convergence.  The finite-difference step
was 0.4, and $10^5$ trajectories were used to calculate one
partial derivative (one component of the gradient), and in total
10 partial derivative were calculated at each step of iterations.
The statistical uncertainty of the
partial derivative was roughly
$5\times 10^{-3}$.  This value is 5 times larger than the
response result, so we observe that the optimal control calculated
by the finite-difference displays larger oscillations
larger than that calculated
by the response formula. We have also compared the result with 
different finite-difference step sizes:  Step 0.2 with
statistical uncertainty $3\times 10^{-3}$ 
and step 1.0 with statistical uncertainty
$1\times 10^{-3}$.
All the results are in good consistency, which means
that the finite-difference method used  in the paper for comparison with
the response theory is reliable.\\

\emph{
On a final note, especially since this is a tribute to Giovanni Ciccotti: this work reminds
me of the subtraction technique (G. Ciccotti and G. Jacucci, Direct computation of dynamical
response by molecular dynamics - mobility of a charged Lennard-Jones particle, Phys. Rev.
Lett., 35, 789792, 1975), with a underlying nonequilibrium/driven dynamics rather than an
equilibrium one. Could the authors comment on the analogy?
}\\

In the paper by Giovanni and Jacucci, the authors use direct
nonequilibrium molecular dynamics simulation to 
calculate the \emph{equilibrium response}, i.e. the mobility of the charged
Lennard-Jones particle, and show that
the nonequilibrium approach has the advantage of smaller statistical error over
the Green-Kubo formula. Our  nonequilibrium simulations,
in comparison, calculate the
\emph{nonequilibrium response}.
As far as the authors concern, the method presented in the paper is new and
has never been studied. The citation and discussion to Giovanni and Jacucci's
work have been added to the manuscript.


\subsection*{The minor scientific comments from referee 1}

We have revised the paper according to the referee's comments. \\

\emph{
(29) page 11, line 28: Do I understand correctly that $\varepsilon$ for the “standard” linear response is
not small at all? The approach then seems doomed from the beginning, and would be
worse and worse as the forcing $u_t$ in the reference driven dynamics increases.
}\\

The answer is yes. We calculate the time-dependent probability density of
the perturbed driven system. If one wants to calculate it  under the
framework of the equilibrium response theory, then the effective perturbation
(to the equilibrium system) is in fact the driving plus the perturbation, as
indicated by the equation on page. 12. Therefore, when the driving 
becomes larger, the effective perturbation to the equilibrium also increases, so the
accuracy of the traditional response theory becomes worse.



\subsection*{The typo corrections from referee 2}

All the typos pointed out by the referee 2 have been corrected in the new manuscript.

\end{document}
