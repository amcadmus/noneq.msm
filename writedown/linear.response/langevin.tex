% \documentclass[aip,jcp,preprint,unsortedaddress,a4paper,onecolum]{revtex4-1}
\documentclass[aip,jcp,a4paper,reprint,onecolumn]{revtex4-1}
% \documentclass[aps,pre,twocolumn]{revtex4-1}
% \documentclass[aps,jcp,groupedaddress,twocolumn,unsortedaddress]{revtex4}

\usepackage[fleqn]{amsmath}
\usepackage{amssymb}
\usepackage[dvips]{graphicx}
\usepackage{color}
\usepackage{tabularx}
\usepackage{algorithm}
\usepackage{algorithmic}

\makeatletter
\makeatother

\newcommand{\recheck}[1]{{\color{red} #1}}
\newcommand{\redc}[1]{{\color{red} #1}}
\newcommand{\bluec}[1]{{\color{blue} #1}}
\newcommand{\vect}[1]{\textbf{\textit{#1}}}
\newcommand{\dd}{\textsf{d}}

\newcommand{\mh}{\mathcal H}
\newcommand{\ml}{\mathcal L}
\newcommand{\mt}{\mathcal T}
\newcommand{\fwg}{{\mathcal A}}
\newcommand{\bwg}{{\mathcal B}}
\newcommand{\EX}{{\textrm{EX}}}
\newcommand{\CG}{{\textrm{CG}}}
\newcommand{\HY}{{\Delta}}



\begin{document}

\title{Linear response theory for core set identification}
\author{Han Wang}
% \affiliation{Institute for Mathematics, Freie Universit\"at Berlin, Germany}
\author{Christof Sch\"utte}
\affiliation{Institute for Mathematics, Freie Universit\"at Berlin, Germany}
% \affiliation{Institute for Mathematics, Freie Universit\"at Berlin, Germany}

\date{\today}

\begin{abstract}
  % In this draft, we are trying to study the core set based Markov
  % State Model (MSM) under non-equilirbium conditions.
\end{abstract}

\maketitle

\section{Theoretical considerations}
\subsection{The linear response theory}
In this sub-section, we give a brief description of the linear response
theory, for details see Ref.~\cite{tuckeman2010statistical} for
example. Here we assume the system is governed by the
perturbed Langevin equation:
\begin{align}\label{eqn:gov-1}
  \dot{\vect q} & = \frac{\partial\mh}{\partial \vect p}
  + \vect C(\vect q,\vect p) F_e(t)\\\label{eqn:gov-2}
  \dot{\vect p} & =- \frac{\partial\mh}{\partial \vect q}
  + \vect D(\vect q,\vect p) F_e(t)
  - \gamma\vect p
  + \sigma\dot{\vect W}
\end{align}
Where $\vect C$ and $\vect D$ are perturbations to the system, and
$F_e(t)$ is the strength of the perturbation as a function of time.
The phase space imcompressibility condition is assumed:
\begin{align}
  \nabla_{\vect q}\cdot\vect C + \nabla_{\vect p}\cdot\vect D = 0
\end{align}
Then the phase space distribution $f(\vect x, t)$ ($\vect x = \{\vect
q, \vect p\}$) is subject to the Fokker-Planck equation:
\begin{align}
  \frac{\partial}{\partial t} f(\vect x, t) - \fwg f(\vect x, t) = 0
\end{align}
where $\fwg$ is the infiniesimal generator gieven by
\begin{align}
  \fwg =
  \frac{\sigma^2}2\Delta_{\vect p}
  - \Big(
  \frac{\partial \mh}{\partial\vect p} + \vect C F_e(t)
  \Big)\cdot\nabla_{\vect q}
  - \Big(
  -\frac{\partial \mh}{\partial\vect q} +
  \vect D F_e(t) - \gamma\vect p
  \Big)\cdot\nabla_{\vect p}
  + 3N\gamma
\end{align}
The equilibrium and unperturbed system
is governed by
the standard Langevin equation, i.e. with
vanishing input perturbation $F_e(t) = 0$ in
Eqn.~\eqref{eqn:gov-1} and \eqref{eqn:gov-2}.
We assume that the perturbation to the
system is so small that both the phase space distribution and the
infiniesimal generator can be viewed as perturbation from the equilibrium
distribution (invariant measure) and the unperturbed operator,
respectively. We have
\begin{align}
  f(\vect x, t) &= f_0(\vect x) + \Delta  f(\vect x, t)
\end{align}
and 
\begin{align}
  \fwg = \fwg_0 + \Delta\fwg(t)
\end{align}
Where the invariant measure satisfies $\fwg_0f_0(\vect x) = 0$,
which is nothing but the Boltzmann distribution
\begin{align}
  f_0(\vect x) \propto e^{-\beta\mh(\vect x)}
\end{align}
The Fokker-Planck equation for the perturbation is therefore
\begin{align}
  \frac{\partial}{\partial t}
  [f_0(\vect x) + \Delta  f(\vect x, t)]
  -
  [\,\fwg_0 + \Delta\fwg(t)\,]
  [\,f_0(\vect x) + \Delta  f(\vect x, t)\,] = 0
\end{align}
The basic assumption for the ``linear response'' is that the high order
term $\Delta\fwg(t)  \Delta  f(\vect x, t)$ in the Fokker-Planck equation
can be neglected. Noticing $\fwg_0 f_0(\vect x) = 0$, we have the
Fokker-Planck
for the leading order of the perturbation:
\begin{align}
  \bigg[
  \frac{\partial}{\partial t}
  - \fwg_0
  \bigg]
  \Delta  f(\vect x, t)
  =
  \Delta\fwg(t) f_0(\vect x)
\end{align}
The operator on l.h.s. is the same as the unperturbed Fokker-Planck equation.
Formally solving this equation with initial condition
$\Delta f(\vect x, 0) = 0$ gives
\begin{align}
  \Delta  f(\vect x, t) =
  \int_0^t \dd s \:
  e^{(t-s)\fwg_0}
  \Delta\fwg(s)f_0(\vect x)
\end{align}
By defining the \emph{dissipative flux},
\begin{align}
  j(\vect x) =
  -\vect C\cdot\nabla_{\vect q}\mh 
  -\vect D\cdot\nabla_{\vect p}\mh
\end{align}
it has been shown that~\cite{tuckeman2010statistical}
\begin{align}
  \Delta\fwg(s)f_0(\vect x) =
  -\beta f_0(\vect x)\, j(\vect x) F_e(s)
\end{align}
For any observation:
\begin{align}\nonumber
  \mathcal O(t)
  &=
  \int\dd \vect x \:O(\vect x)f(\vect x, t)  \\\nonumber
  &=
  \int\dd \vect x\, O(\vect x)\,
  [f_0(\vect x) + \Delta f(\vect x, t)] \\\nonumber
  &=
  \langle O\rangle_0
  -
  \beta
  \int \dd \vect x\:
  O(\vect x)
  \int_0^t\dd s\,
  e^{(t-s)\fwg_0}
  f_0(\vect x)\, j(\vect x)F_e(s)  \\\nonumber
  &=
  \langle O\rangle_0
  -
  \beta
  \int_0^t\dd s\,
  F_e(s)
  \int \dd \vect x\,
  O(\vect x)\,
  e^{(t-s)\fwg_0}
  f_0(\vect x) j(\vect x)  \\
  &=
  \langle O\rangle_0
  -
  \beta
  \int_0^t\dd s\,
  F_e(s)
  \int \dd \vect x\,
  f_0(\vect x) j(\vect x)\,
  e^{(t-s)\bwg_0}
  O(\vect x)\\
  &=
  \langle O\rangle_0
  -
  \beta
  \int_0^t\dd s\,
  F_e(t-s)
  \int \dd \vect x\,
  f_0(\vect x) j(\vect x)\,
  e^{s\bwg_0}
  O(\vect x)
\end{align}
Where $\bwg_0$ is the infiniesimal generator of the \emph{unperturbed} backward
Kolmogorov equation, which is the adjoint operator of $\fwg_0$.
We have
\begin{align}
  e^{s\bwg_0} O(\vect x) = \mathbb E_{\vect x} [O(\vect X_s)]
\end{align}
where $\vect X_t$ is the trajectory of the unperturbed Langevin equation
starting at $\vect x$.
Therefore, we have
% Notice that $e^{i\ml_0(t-s)}O(\vect x) = O(\phi_{t-s}(\vect x))$,
% where $\phi_t(\vect x)$ is the flow mapping of the \emph{unperturbed}
% Hamiltonian system. We have
\begin{align}
  \mathcal O(t)
  =
  \langle O\rangle_0
  -
  \beta
  \int_0^t\dd s\,
  F_e(t-s)
  \int \dd \vect x\,
  f_0(\vect x)
  j(\vect x)\,
  \mathbb E_{\vect x} [O(\vect X_s)]
\end{align}
or equivalently:
\begin{align}\label{eqn:lr}
  \mathcal O(t)
  =
  \langle O\rangle_0
  -
  \beta
  \int_0^t\dd s\,
  F_e(t-s)
  \langle
  j(0)\,
  \mathbb E_{\vect x} [O(\vect X_s)]
  \rangle_0
\end{align}
where $  \langle
  j(0)\,
  \mathbb E_{\vect x} [O(\vect X_s)]
  \rangle_0$ is the \emph{equilibrium time
  correlation} between the observation $O$ at time $s$ and the
dissipative flux $j$ at time 0.


\subsection{The Core set as an observation of the system}

The core set plays a very important role in the MSM theory. One way to
identify core set is to consider the ``stability'' of the invariant
measure $f_0$ under some mapping $\mt_\alpha$ of time step $\alpha$:
\begin{align}
  C_\alpha = \big\{
  \vect x \,\vert\, \mt_\alpha f_0(\vect x) > f_0(\vect x)
  \big\}
\end{align}
Generally, it is impossible and not necessary to consider the core set
in the $3N$ dimensional configuration space. For example, a
macromolecule solving in water, the DOFs of water is not of interest
at all. Therefore, we consider some phase space compressing function
$\vect y = \vect y(\vect x)$, with $\vect y \in \Omega \subset \mathbb
R^M,\ M\ll 3N$. We call $\Omega$ the observation space.  The projected
invariant measure on the observation space is therefore
\begin{align}
  \mathcal P f_0 (\vect y)
  =
  \int
  \dd \vect x\,
  f_0(\vect x) \,
  \delta (\vect y - \vect y(\vect x)) 
\end{align}
The core set on the observation space is
\begin{align}
  \widetilde
  C_\alpha = \big\{
  \vect y \,\vert\,
  \mathcal P\mt_\alpha f_0(\vect y) > \mathcal P f_0(\vect y)
  \big\}
\end{align}
% with $\widetilde \mt_\alpha$ defined by
% $\widetilde\mt_\alpha = \mathcal P\circ\mt_\alpha$, namely
% \begin{align}
%   \widetilde\mt_\alpha\, p_0(\vect y)
%   =
%   \mathcal P\circ\mt_\alpha f_0 (\vect y)
%   =
%   \int
%   \dd \vect x\,
%   [\mt_\alpha f_0](\vect x) \,
%   \delta (\vect y - \vect y(\vect x)) 
% \end{align}
For convenience, we define $\Delta \mt_\alpha =
\mt_\alpha - \mathcal I$, where $\mathcal I$ is the
identity. If $[\mathcal P\circ\Delta \mt_\alpha f_0](\vect y) > 0$, then
$\vect y$ is in the core set $\widetilde C_\alpha $.
\begin{align}\nonumber
  [\mathcal P\circ\Delta \mt_\alpha f_0](\vect y)
  &=
  \int
  \dd \vect x\,
  \Delta \mt_\alpha f_0(\vect x) \,
  \delta (\vect y - \vect y(\vect x))  \\\nonumber
  &=
  \lim_{\sigma \rightarrow 0}
  \int
  \dd \vect x\,
  \Delta \mt_\alpha f_0(\vect x) \,
  K_{\sigma} (\vect y - \vect y(\vect x))  \\\nonumber
  &=
  \lim_{\sigma \rightarrow 0}
  \int
  \dd \vect x\,
  \Delta \mt_\alpha f_0(\vect x) \,
  K_{\sigma, \vect y} (\vect x)  \\\nonumber
  &=
  \lim_{\sigma \rightarrow 0}
  \int
  \dd \vect x\,
  f_0(\vect x) \,
  [\Delta \mt_\alpha]^\dagger K_{\sigma, \vect y} (\vect x)\\
  &=
  \lim_{\sigma \rightarrow 0}\,
  \langle
  [\Delta \mt_\alpha]^\dagger K_{\sigma, \vect y}
  \rangle_0
\end{align}
where $[\Delta \mt_\alpha]^\dagger$ is the adjoint operator of $\Delta
\mt_\alpha$, which is actually $\Delta \mt_{-\alpha}$. The delta
function is replaced by the limit of a mollifier $K_\sigma(\vect y)$,
which is infinitely smooth, compact supported of size $\sigma$,
bounded and $\int
K_\sigma(\vect y)\dd \vect y = 1$.  The physical meaning is that we
can consider the expectation value of the stability of the mollifier
$K_{\sigma, \vect y}$ instead of the stability of the invariant measure.
This provides the convenience of using the linear response theory showed
in the previous section. By Eqn.~\eqref{eqn:lr},
\begin{align}\nonumber
  [\mathcal P\circ\Delta \mt_\alpha f_t](\vect y)
  &=
  \lim_{\sigma \rightarrow 0}\,
  \langle
  [\Delta \mt_\alpha]^\dagger K_{\sigma, \vect y}
  \rangle_t \\
  &=
  \lim_{\sigma \rightarrow 0}\,
  \Big\{
  \langle
  [\Delta \mt_\alpha]^\dagger K_{\sigma, \vect y}
  \rangle_0  -
  \beta
  \int_0^t\dd s\,
  F_e(t-s)
  \langle
  j(0)\,
  \mathbb E_{\vect x}\{
  [\Delta \mt_\alpha]^\dagger K_{\sigma, \vect y}(s)\,
  \}
  \rangle_0
  \Big\}
\end{align}
Therefore, the core set at time $t$ is defined by
\begin{align}
  \widetilde C_\alpha^t =
  \Big\{
  \vect y
  \,\Big\vert\,
  [\mathcal P\circ\Delta \mt_\alpha f_t](\vect y) > 0
  \Big\}
\end{align}


\section*{References}
\bibliography{ref}{}
\bibliographystyle{unsrt}





\end{document}